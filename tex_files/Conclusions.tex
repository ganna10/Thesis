This work attempted to determine the important processes influencing ozone production through a series of detailed modelling studies.
These studies looked at how VOC degradation specified by different chemical mechanisms, varying the input of NMVOC speciation and quantified the direct effects of temperature on ozone production.
Box modelling experiments were designed and repeated using different chemical mechanisms that are used by the modelling community for global and regional modelling.
Reduced chemical mechanisms typically used by modelling groups for regional and global studies were chosen for this study and compared to the highly detailed MCM chemical mechanism.

The first study looking at the effects of different simplification techniques used by chemical mechanisms on ozone production using a tagging technique that allowed attribution of ozone production to emitted VOC.
The chemical mechanisms and model runs were harmonised in such a way that any differences in ozone production between the chemical mechanisms was due to their representations of VOC degradation.
Despite these harmonisations, the difference in the ozone maximum on the first day was $21$?~ppbv when including the outlier RACM chemical mechanism and $15$?~ppbv when not including RACM.

\ce{O_x} production budgets allocated to emitted VOC showed larger differences between chemical mechanisms with VOC represented by lumped mechanism species than explicitly represented VOC.
In particular, the representation of aromatic VOC consistently produced lower \ce{O_x} in the reduced chemical mechanisms and was the source of the low ozone mixing ratios in the RACM chemical mechanism.
Hence, when modelling urban areas with significant emissions of aromatic VOC, RACM chemistry may underpredict ozone production.

The lumped intermediate chemical mechanism, CRI~v2, produced the most similar amounts of \ce{O_x} from each VOC to the MCM.
The lumped intermediate species used by the CRI~v2 were developed to produce similar amounts of ozone as the MCM

The largest differences in ozone production from the VOC was after the first day of simulations where ozone production was fuelled by the degradation of less-reactive alkanes.
The peak in ozone production from alkane species in lumped-molecule (MOZART-4, RADM2, RACM, RACM2) and lumped-structure (CBM-IV, CB05) chemical mechanisms was lower than the MCM chemical mechanisms.

How do the simplification techniques used by different chemical mechanisms affect ozone production?
Which processes are responsible for differences in ozone production with different chemical mechanisms?

What is the influence on modelled ozone production when using different speciations of emitted NMVOCs?

Does this influence change when using different chemical mechanisms?

Are temperature-dependent emissions or chemical processes more important for ozone production with increasing temperature? 

How is the ozone-temperature relationship treated by different chemical mechanisms?

Overall:
How do the simplification techniques used by different chemical mechanisms affect ozone production?

What are the most important chemical processes when simulating ozone production?
