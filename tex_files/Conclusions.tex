The detailed box modelling studies of Chap.~\ref{c:papers} were designed to address the research questions of Sect.~\ref{s:research_questions}.
These studies compared the effects on ozone production from VOC degradation specified by various chemical mechanisms, varying the input of NMVOC speciation and quantified the effects of temperature with different \ce{NO_x} conditions.
In each study the near-explicit MCM~v3.2 was the reference chemical mechanism with the simulations repeated using reduced chemical mechanisms typically used by modelling groups for regional and global studies.

The first study determined the effects of different simplification techniques used by chemical mechanisms on ozone production using a tagging technique that attributed ozone production to emitted VOC.
Varying the representation of VOC degradation chemistry in the box model led to differences in the first day peak ozone of $21$~ppbv when including the outlier RACM chemical mechanism and $8$~ppbv when not including RACM.
Moreover, only CRI~v2 and RADM2 led to higher ozone mixing ratios than the MCM~v3.2 on the first two days of model simulations.

\ce{O_x} production budgets allocated to emitted NMVOC showed larger differences between chemical mechanisms with VOC represented by lumped mechanism species than explicitly represented NMVOC.
In particular, the representation of aromatic VOC consistently produced lower \ce{O_x} in the reduced chemical mechanisms and was the source of the low ozone mixing ratios in the RACM chemical mechanism.
Hence, when modelling urban areas with significant emissions of aromatic VOC, RACM chemistry may underpredict ozone production.
Thus the choice of chemical mechanism influenced the amount of ozone produced from the box model simulations.

The lumped intermediate chemical mechanism, CRI~v2, produced the most similar amounts of \ce{O_x} from each VOC to the MCM.
The lumped intermediate species used by the CRI~v2 were developed to produce similar amounts of ozone as the MCM.
Thus the technique of lumping intermediate species rather than lumping VOCs appears very promising for representing ozone production.

Production of \ce{O_x} from reactive NMVOC, such as alkenes, using both lumped-molecule (MOZART-4, RADM2, RACM, RACM2) and lumped-structure (CBM-IV, CB05) chemical mechanisms was generally similar to that using the MCM~v3.2.
Peak \ce{O_x} production from alkane species in lumped-molecule and lumped-structure chemical mechanisms was lower than peak \ce{O_x} production with the MCM~v3.2.
As alkanes are less-reactive VOC, they are more likely to be transported downwind of emission sources affecting ozone production in urban background areas.
Thus an underestimation in the ozone production from alkanes in reduced chemical mechanisms may impact on simulated background ozone levels.

The representation of NMVOC by lumped-molecule or lumped-structure species in the reduced chemical mechanisms led to generally lower ozone production than the MCM~v3.2 with the emitted NMVOC.
Moreover, the secondary degradation chemistry specified by the reduced chemical mechanisms for many NMVOC led to a faster breakdown of the emitted species ultimately leading to lower \ce{O_x} production from the emitted species.

The second study looked at the influence of varying the speciation of NMVOC emissions from the solvent sector on ozone production.
The simulations were performed using the MCM~v3.2 as the reference chemical mechanism and MOZART-4 and RADM2 were reduced chemical mechanisms.
Differences of up to $9$~ppbv were obtained between solvent sector EIs using a single chemical mechanism.
Difference in peak ozone of up to $15$~ppbv were obtained from box model simulations using different solvent sector EIs and the same chemical mechanism.
Including NMVOC emissions from all other emissions sectors and varying the NMVOC emissions from the solvent sector, reduced the differences in peak ozone mixing ratios (up to $9$~ppbv).
Further including emissions from biogenic sources while varying the NMVOC emissions from the solvent sector resulted in lower still differences in ozone mixing ratios while using the same chemical mechanism (up to $8$~ppbv).
Thus the choice of solvent sector EI and chemical mechanism influenced ozone production.

First day production of \ce{O_x} was controlled by the contribution of reactive NMVOC such as alkenes, however the emission of large quantities of alkenes from the solvent sector is debatable.
The cumulative \ce{O_x} production at the end of the seven days was influenced by the \ce{O_x} production from less-reactive NMVOC such as alkanes and oxygenated NMVOC.
The contribution of alkanes specified by the EIs had a positive correlation with cumulative ozone production while the specified contributions of oxygenated NMVOC had a negative correlation with ozone production.
Furthermore, EIs specifying larger emissions from alkanes produced the highest ozone while EIs specifying larger emissions of oxygenated NMVOC produced the lowest amounts of \ce{O_x}.
The EIs of the same country at different time points (Greece 1995 and 2005, United Kingdom 1998 and 2008) indicated that the more recent EI specified more emissions of oxygenated VOC than alkanes.
Thus when preparing updated or new EIs the amount of alkane and oxygenated species are particularly important with respect to ozone production.

Changing the speciation of an EI within the box model setup influenced the ozone production with these differences reproduced when using different chemical mechanisms.
When considering instantaneous ozone production, alkenes contributed most to ozone production while cumulative ozone production after seven days was influenced by alkanes and oxygenated NMVOC.
As a first scoping study as to whether EIs need to be updated do reduce modelling uncertainity, the box modelling study indicates that the differences in solvent sector speciation do influence ozone production.
However, these differences may differ when using more complex 3D models and further work is warranted to determine whether different EI speciations influence ozone production under different conditions.

The final study looked at the relationship of ozone with temperature using different \ce{NO_x}-conditions roughly over central European conditions.
In order to verify whether temperature dependent increases in reaction rates or increases in isoprene emissions from nature are more important for ozone with temperature, separate simulations using a temperature-independent and then temperature-dependent source of isoprene emissions were performed.
Simulations were performed using the MCM~v3.2, CRI~v2, RADM2, MOZART-4 and CB05 chemical mechanisms to determine whether this relationship differed between representations of atmospheric chemistry.

Each chemical mechanism reproduced the non-linear relationship of ozone with temperature and \ce{NO_x}.
Increases in ozone with temperature due to faster reaction rates was more important in urban areas than the increase in isoprene emissions with temperature regardless of \ce{NO_x} conditions.
The largest increases were when using moderate \ce{NO_x} emissions and the slowest increase was with low-\ce{NO_x} conditions.

Differences in the RADM2 representation of isoprene degradation to the MCM and other chemical mechanisms led to a lower increase in ozone with temperature than the other chemical mechanisms.
This lower increase in ozone from isoprene degradation in RADM2 was due to the RADM2 not representing the well-known degradation products of isoprene degradation, methacrolein and methyl vinyl ketone.
The lower ozone production from isoprene with RADM2 could lead to underestimations in the ozone production in modelling studies with significant amounts of isoprene emissions, which is in fact the case for many areas as isoprene is the VOC with the most emissions on the global scale and many remote regions are estimated to have high isoprene emissions.

The faster oxidation of VOC with temperature caused the faster increase in ozone with temperature.
The faster reaction rates was mainly due to the increase in OH with temperature which is coupled to the increase in ozone with temperature.

Comparing the box model simulations to observational and 3-D model output showed that the rate of change in ozone with temperature in tbe box model was about half the rate of change of ozone with temperature using both observational and 3-D model data.
Also the spread in the ozone-temperature curves from simulations using the different chemical mechanisms was insignificant compared to the spread in observational and 3-D model output.
The best boxmodel simulations were when using a temperature-dependent source of isoprene emissions and high-\ce{NO_x} conditions.
Thus modelling studies should include a temperature-independent source of isoprene emissions.

The box model experiments were setup to consider instanteous ozone production whereas observations and 3-D models also include data representing stagnant conditions.
These atmospheric conditions results from low-wind speeds and high temperature leading to the previous day's ozone not being transported away from the region and the current day's ozone production is added on top of that.

The ozone-temperature relationship across different \ce{NO_x} regimes has not been extensively studied in regional or global modelling studies.
