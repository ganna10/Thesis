The research questions of Sect.~\ref{s:research_questions} were addressed by the box modelling studies of Chap.~\ref{c:papers} and these questions are answered in this chapter.
In each study the near-explicit MCM~v3.2 was the reference chemical mechanism with the simulations repeated using reduced chemical mechanisms typically used by modelling groups for regional and global studies.

The first study compared the effects of different simplification techniques used by chemical mechanisms on ozone production.
The reduced chemical mechanisms used in the comparison were developed through lumping VOC degradation intermediates (CRI~v2), aggregating emitted NMVOCs into lumped molecules (MOZART-4, RADM2, RACM, RACM2) and expressing the emitted NMVOCs by lumped structure species (CBM-IV, CB05).
Out of these three simplification techniques, the lumped-intermediate approach produced the most similar amounts of ozone from each VOC to the MCM~v3.2 while the lumped-molecule and lumped-structure approaches generally produced less ozone than the MCM~v3.2 from the degradation of the VOC.
Thus the technique of lumping intermediate species rather than lumping VOCs appears promising for representing ozone production in future chemical mechanisms.

The second objective of the first study was to determine the processes responsible for the differences in ozone production using the different reduced chemical mechanisms.
The largest differences in ozone production from the reduced chemical mechanisms to the MCM~v3.2 were obtained for VOC represented by lumped mechanism species than those VOC explicitly represented.
In particular, the representation of aromatic VOC consistently produced lower ozone in all the reduced chemical mechanisms.
These differences in ozone production from aromatic VOC are not surprising as the degradation of aromatic VOC is an area of uncertainty \citep{Atkinson:2003}.
Even the detailed degradation chemistry of aromatic VOC in the MCM~v3.2 was unable to reproduce the results from chamber experiments \citep{Bloss:2005}.

Another process leading to lower ozone production from VOC degradation than the MCM~v3.2 was the faster break down of emitted VOC into smaller sized degradation products by lumped-molecule and lumped-structure chemical mechanisms.
In particular, the faster break down of alkanes in lumped-molecule and lumped-structure chemical mechanisms led to a lower peak of ozone production than the MCM~v3.2.
As alkanes are less-reactive VOC, they are more likely to be transported downwind of emission sources affecting ozone production in urban background areas.
Thus underestimating the ozone production from alkanes may impact simulated ozone levels of urban background areas.

The second study looked at the influence of varying the speciation of NMVOC emissions from the solvent sector on ozone production.
In box model experiments using the MCM~v3.2, differences in ozone production resulted from the different speciations of NMVOC emissions.
Ozone production on the first day was influenced by the amounts of alkenes and aromatic VOC specified by the solvent sector emission inventory (EI).
While the contribution of alkanes and oxygenated VOC by the solvent sector EI determined ozone production at the end of the simulations.

Similar differences in ozone production were obtained when repeating the box model simulations using reduced chemical mechanisms (MOZART-4, RADM2).
Thus although the emissions of NMVOC were represented by much fewer species in the reduced chemical mechanisms than the MCM~v3.2, the differences in ozone production were not dampened by reducing the complexity of the chemical mechanism.

This study was designed as a scoping study to determine whether updating the speciation of an EI influenced model predictions.
Given the results of this initial study, further modelling studies using different atmospheric conditions and more complex models are warranted to obtain a complete picture of how varying the speciation of NMVOC emissions influences modelled ozone levels.

The final study modelled the relationship between ozone, temperature and \ce{NO_x} using central European conditions.
In order to verify whether temperature-dependent increases in reaction rates or isoprene emissions from nature are more important for the increase in ozone with temperature, separate simulations using a temperature-independent and temperature-dependent source of isoprene emissions were performed.
From these simulations, the absolute increase in ozone with temperature due to faster reaction rates was slightly higher than the increase in ozone with temperature due to increased isoprene emissions with temperature regardless of \ce{NO_x} conditions.
This result was surprising as studies (e.g. \citep{Racherla:2008}, \citep{Doherty:2013}) have consistently attributed the increase of ozone with temperature to increased isoprene emissions from vegetation.

The increase in ozone with temperature in all \ce{NO_x} conditions was principally due to the faster loss rates of the emitted VOC with temperature as the OH-reactivity of the emitted VOC increased with temperature.
As expected, peroxy nitrate (\ce{RO2NO2}) chemistry also played a role in the increase of ozone production with temperature with increased \ce{RO2NO2} decomposition at higher temperatures leading to more \ce{RO2} available to produce ozone.
Thus the initial oxidation of emitted VOC and \ce{RO2NO2} chemistry are critical to modelling the relationship between ozone, temperature and \ce{NO_x}.

Simulations were performed using reduced chemical mechanisms (CRI~v2, RADM2, MOZART-4, CB05) to determine whether the relationship between ozone, temperature and \ce{NO_x} differed between representations of atmospheric chemistry.
Each chemical mechanism reproduced the non-linear relationship between ozone, temperature and \ce{NO_x} with the choice of chemical mechanism not significantly changing this relationship.
The rate of increase of ozone with temperature was found to be more sensitive to the amount of mixing rather than the choice of chemical mechanism after comparing the box model simulations to observational and 3-D model output.

Overall, the representation of detailed atmospheric chemistry influenced ozone production as in each of the studies differences between ozone levels were obtained when repeating model simulations with different chemical mechanisms.
The representation of emitted NMVOC by the different chemical mechanisms was critical leading to differences in ozone production in each study.
The secondary degradation processes of \ce{RO2NO2} chemistry and the rate of break down of emitted NMVOC by reduced chemical mechanisms also had implications for ozone production.
