This chapter details the model set-up in Sect.~\ref{ss:model_setup}, initial and boundary conditions are described in Sect.~\ref{s:initial_conditions} used to address the research questions of this work (Sect.~\ref{s:research_questions}).

\section{Air Quality Modelling} \label{s:modelling}
AQ models are mathematical representations of the atmosphere designed to produce continuous output fields that aid in explaining sources of air pollution.
All models numerically solve the system of differential equations describing the conservation of chemical species \citep{Russell:2000}.

Eulerian models are the most common type of AQ model \citep{Russell:2000}.
These models use fixed grid-boxes where species are transported in and out of the boxes to describe the atmosphere \citep{Seinfeld:2006}. 
Box models are the simplest type of a model (zero-dimensional) having uniform atmospheric concentrations that are only a function of time.
Whereas 3-D models describe atmospheric concentrations as a function of time, latitude, longitude and height thus requiring much more computing power than a box model \citep{Seinfeld:2006}.
Box models lack realism but are useful for studying the detailed processes that influence air quality.

Solving the system of differential equations requires initial and boundary conditions for each chemical species.
Initial conditions fix the starting concentrations of each species in each grid-box.
Boundary conditions require knowledge of the concentration and transport of each species at the boundary edges of the model grid.

\subsection{Model Description and Setup} \label{ss:model_setup}
The MECCA (Module Efficiently Calculating the Chemistry of the Atmosphere) box model was used throughout this work.
MECCA was developed by \citet{Sander:2005} and adapted to include MCM~v3.1 chemistry by \citet{Butler:2011}.
The studies of \citet{Kubistin:2010} and \citet{Lourens:2016} used MECCA.

MECCA is written in Fortran and runs on UNIX/Linux platforms.
The Kinetic Pre-Processor (KPP, \citet{Damian:2002}) was used to process the chemical mechanism and generate Fortran code further compiled within MECCA.
KPP has many choices of numerical solver for solving the differential equations, this work used a Rosenbrock solver (the ros3 option).

All fluxes of the chemical species into and out of the box are handled by KPP.
The chemical mechanism file includes pseudo-unimolecular reactions specifying the emissions and dry deposition of chemical species with the relevant rate.
The chemical species that are emitted into the model and the emission rates are read into the model using a namelist file.  
Namelist files were also used to specify the initial and boundary conditions of chemical species.

\begin{table}[t]%
    \begin{center}%
        \caption{General settings used for MECCA box model in this study.}%
        \begin{tabular}{ll}%
            \hline \hline
            \textbf{Model Parameter} & \textbf{Setting} \\
            \hline \hline
            Pressure & $1013$ hPa \\
            Relative Humidity & $81$ \% \\
            Starting Date and Time & 27th March 06:00 \\
            Model Time Step & $20$ mins \\
            \hline \hline
        \end{tabular}%
        \label{t:model_setup}%
    \end{center}%
\end{table}%
The physical parameters used in MECCA throughout this work are detailed in Table~\ref{t:model_setup}.
In the first two studies, temperature was held constant at $293$~K and the boundary layer height was fixed at $1000$~m.
In the final study, MECCA was updated to include a diurnal boundary layer height taken from the BAERLIN2014 campaign \citep{Bonn:2016}.
Also, in the final study temperature was systematically varied between $288$ and $313$~K ($15$--$40$~\degree C).
These changes to the model setup for the final study are outlined in Chap.~\ref{c:paper_3}.

Photolysis rates in were paramaterised as a function of the solar zenith angle based on the approach of the MCM \citep{Jenkin:1997}.
This paramaterisation utilises the latitude and in the first two studies $34$~\degree N, roughly the city of Los Angeles, was used.
In the final study, the latitude was set to $51$~\degree N simulating central European conditions.

\section{Chemical Mechanisms} \label{s:chemical_mechanisms}
\begin{table}[t]%
    \begin{center}%
        \caption{Chemical mechanisms used in the study.}%
        \scalebox{.85}[.85]{\begin{tabular}{lll}%
                \hline \hline
                \textbf{Chemical Mechanism} & \textbf{Lumping Type} & \textbf{Reference} \\
                \hline \hline
                \multirow{3}{*}{MCM v3.1 and v3.2} & \multirow{3}{*}{No lumping} & \citet{Jenkin:1997}, \citet{Jenkin:2003} \\
                & & \citet{Saunders:2003}, \citet{Bloss:2005} \\
                & & \citet{MCM_Site} \\
                CRIv2 & Lumped intermediate & \citet{Jenkin:2008} \\
                MOZART-4 & Lumped molecule & \citet{Emmons:2010} \\
                RADM2 & Lumped molecule & \citet{Stockwell:1990} \\
                RACM & Lumped molecule & \citet{Stockwell:1997} \\
                RACM2 & Lumped molecule & \citet{Goliff:2013} \\
                CBM-IV & Lumped structure & \citet{Gery:1989} \\
                CB05 & Lumped structure & \citet{Yarwood:2005} \\
                \hline \hline
            \end{tabular}%
        }%
        \label{t:mechanisms}%
    \end{center}%
    \vspace{-13mm}
\end{table}%
The first part of this study compared the influence of different chemical mechanisms on ozone produced.
The chemical mechanisms used in this study are listed in Table~\ref{t:mechanisms} and descriptions of these chemical mechanisms are found in Paper~I (Sect.~\ref{s:chemical_mechanism_results}).

The Master Chemical Mechanism (MCM, \citet{Jenkin:1997, Jenkin:2003, Saunders:2003, Bloss:2005, MCM_Site}) is a near-explicit chemical mechanism and this high level of detail made it ideal as the reference chemical mechanism in each study of this work.
The Common Representative Intermediates (CRI) chemical mechanism \citep{Jenkin:2008} is an lumped intermediate mechanism where the degradation productions are aggregated (lumped) rather than primary VOC.
Lumped molecule chemical mechanisms aggregate primary VOC into mechanism species and is the most simplification common technique.
The lumped-molecule chemical mechanisms used in this work were Model for OZone and Related chemical Tracers (MOZART, \citet{Emmons:2010}), Regional Acid Deposition Model (RADM2, \citet{Stockwell:1990}), Regional Atmospheric Chemistry Mechanism (RACM, \citet{Stockwell:1997}) and RACM2 \citet{Goliff:2013}.
The Carbon Bond mechanisms CBM-IV \citep{Gery:1989} and CB05 \citep{Yarwood:2005} were the lumped-structure chemical mechanisms used in this work.
These chemical mechanisms were chosen as they are commonly used by the AQ modelling community as outlined by the review of European modelling groups by \citet{Baklanov:2014}.


\section{Implementing Chemical Mechanisms in MECCA} \label{s:mechanisms_MECCA}
Each chemical mechanism listed in Table~\ref{t:mechanisms} was adapted to the KPP format for use in the MECCA box model.
The WRF-Chem model \citep{Grell:2005} includes KPP versions of RADM2, RACM and CBM-IV and was the starting point for using these chemical mechanisms in MECCA.
The full version of the CRI~v2 was obtained from \mbox{\url{http://mcm.leeds.ac.uk/CRI}} while for all other chemical mechanisms the original reference was used.

In order to focus on the differences in the representation of VOC degradation between the chemical mechanisms, a number of harmonisations between the chemical mechanisms were implemented.
For these harmonisations, the approaches used by the reference chemical mechanism (MCM~v3.2) were implemented in the reduced chemical mechanisms,
These changes are detailed in the supplementary material of the first paper of this thesis (Chap.~\ref{c:paper_1}). 
The main results from the chemical mechanism comparison study are presented in Sect.~\ref{s:chemical_mechanism_results}.

\section{Initial and Boundary Conditions} \label{s:initial_conditions}
In all simulations through this work, methane (\ce{CH4}) was fixed to $1.75$~ppmv while carbon monoxide (CO) and \ce{O3} are initialised at $200$~ppbv and $40$~ppbv and then allowed to evolve freely.
The initial conditions for NMVOCs were held constant until noon of the first day of simulations to simulate a fresh plume of emissions.

The initial conditions for NMVOC species differed in each experiment, a brief summary is given below and details are found in the respective publications (Chap.~\ref{c:paper_1}, Chap.~\ref{c:paper_2} and Chap.~\ref{c:paper_3}).
The first study applied the tagging approach introduced in \citet{Butler:2011} (discussed further in Sect.~\ref{s:tagging}) to different chemical mechanisms.
The initial conditions of the Los Angeles experiments in \citet{Butler:2011} were used in the MECCA set-up with MCM~v3.2 chemistry to determine the emissions needed for constant mixing ratios of the NMVOCs.
These emissions were mapped to the appropriate chemical species of each chemical mechanism in Table~\ref{t:mechanisms} keeping amount of emitted NMVOC constant between model setups.

\begin{table}[t]%
    \begin{center}%
        \caption{The solvent sector emission inventories compared in this study.}%
        \begin{tabular}{lllP{5.2cm}}%
            \hline \hline
            \textbf{Speciation} & \textbf{Comment} & \textbf{Reference} \\ 
            \hline \hline
            TNO & European average &  \citet{Builtjes:2002} \\ \hline
            IPCC & Model Specific & \citet{Ehhalt:2001} \\ \hline
            EMEP & Model Specific & \citet{Simpson:2012} \\ \hline
            DE94 & Country Specific & \citet{Friedrich:2002} \\ \hline
            GR95 & Country Specific & \citet{Sidiropoulos:2007} \\ \hline
            GR05 & Country Specific & \citet{Sidiropoulos:2007} \\ \hline
            UK98 & Country Specific & \citet{Goodwin:2000} \\ \hline
            UK08 & Country Specific & \citet{Murrells:2010} \\ 
            \hline \hline
        \end{tabular}%
        \label{t:solvent_speciations}%
    \end{center}%
    \vspace{-7mm}
\end{table}%
The second study compared the ozone production using the emission inventories for the solvent sector listed in Table~\ref{t:solvent_speciations}.
This study was designed to simulate a theoretical urban area of $1000$~km$^2$ with total NMVOC emissions of $1000$~tons/day.
The solvent sector contributes $\sim43$~\% by mass of total emissions \citep{AQEU:2011}, thus total NMVOC emissions of $430$~tons/day were used.

The total NMVOC emissions of the solvent sector were mapped to MCM~v3.2 species based on the speciations of each emission inventory.
Model simulations were repeated using MOZART-4 and RADM2 to investigate whether changing the chemical mechanism affects the differences in ozone concentrations between the solvent sector emission inventories.

The final study looked at the ozone-temperature relationship over central Europe and the emissions of NMVOC over Benelux (Belgium, Netherlands and Luxembourg) were used.
The TNO\_MACCIII emissions for the year 2011 were used as anthropogenic NMVOC emissions and mapped to MCM~v3.2 species.
Temperature indepedent emissions of biogenic species (isoprene and monoterpenes) were taken from the EMEP speciation \citep{Simpson:2012}.
Simulations using temperature-dependent emissions of isoprene used the MEGAN2.1 \citep{Guenther:2012} algorithm.
All simulations were repeated using the CRI~v2, MOZART-4, RADM2 and CB05 chemical mechanisms.

\ce{NO_x} conditions generating VOC-and-\ce{NO_x} sensitive chemistry were used in the first two studies.
This was achieved by emitting the amount of NO required to balance the source of radicals at each time step.
While the final study assessed the relationship between ozone and temperature with different \ce{NO_x} conditions.
For these simulations, a constant source of NO emissions was systematically varied between $5.0 \times 10^9$ and $1.5 \times 10^{12}$~molecules~(NO)~cm$^{-2}$~s$^{-1}$.

No chemical boundary conditions were used in the first studies as the experiment was setup as a contained box.
In the final study, MECCA included a diurnal profile of the PBL with vertical mixing into the free troposphere.
The boundary conditions for the free troposphere mixing ratios for \ce{O3}, \ce{CH4} and CO were set to $50$~ppbv, $1.8$~ppmv and $116$~ppbv respectively. 
These mixing ratios were taken from the $700$~hPa height using the MATCH-MPIC chemical weather forecast data (\url{http://cwf.iass-potsdam.de/}) from March~21st.

\section{Tagging of Chemical Mechanisms} \label{s:tagging}
AQ models can be used to allocate the effects of different precursors or emission sources on ozone production.
For example, source removal studies perform separate model simulations with and without emissions from a sector to quantify the effect of the sector on ozone production, such as the quantification of megacity emissions on ozone production n \citet{Butler:2009}.
Tagging is another approach where the chemical mechanism includes additional chemical species labelled (tagged) with source information.
For example, \citet{Emmons:2012} updated MOZART-4 chemistry to attribute ozone production to emission sources of \ce{NO_x} emissions.

In \citet{Butler:2011}, tagged NMVOC chemistry allows allocation of ozone production to emitted NMVOC.
This tagging approach considers \ce{O_x} production as a proxy for \ce{O3} production, this assumption is only valid for \ce{NO_x}-limited and VOC-and-\ce{NO_x} sensitive chemistry not in high \ce{NO_x} conditions.
The \ce{O_x} family includes \ce{O3}, \ce{NO2}, \ce{O(^1D)}, \ce{O(^3P)}, \ce{NO3}, \ce{N2O5} and other species involved in fast production and loss cycles with \ce{NO2}.

All chemical mechanisms in Table~\ref{t:mechanisms} were tagged using the approach of \citet{Butler:2011}.
In the first study, the tagging approach was the basis for comparing the respresentations of VOC degradation chemistry and their effects on ozone production.
The second study used VOC-and-\ce{NOx}-sensitive conditions and simulations using the tagged chemical mechanisms were a useful tool in determining the sources of differences in ozone production from the solvent sector emission inventories of Table~\ref{t:solvent_speciations}.
The variable \ce{NO_x} conditions used in the third study meant that using the tagging approach was not possible.
Thus all model simulations assessing the ozone-temperature relationship with different \ce{NO_x} conditions were performed with non-tagged versions of the chemical mechanisms.
