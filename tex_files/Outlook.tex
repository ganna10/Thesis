Ozone pollution is a major problem in Europe and there is currently no legally binding limit value for ozone in Europe.
Currently, the EU Directive 2008/50/EG sets a target value for human health requires the mean eight hourly ozone concentration not exceeding $120$~$\mu$g~m$^{-3}$ (= $60$~ppbv) on more than $25$ calendar days.
The same EU Directive also sets an AOT40 target value for the exposure of vegetation to ozone of less than $18,000$~$\mu$g~m$^{-3} \cdot$~h, where AOT40 is the sum of the differences between the mean hourly ozone values above $80$~$\mu$g~m$^{-3}$ (= $40$~ppbv) and the value of $80$~$\mu$g~m$^{-3}$ between 8am and 8 pm from May to July.

The idea of a legally binding limit value for ozone pollution by the EU is one that should be taken into consideration for the well-being of all people and vegetation.
Having such a limit value that countries can realistically meet is a difficult task and one that tasks current modelling groups and their models.
AQ modelling groups would be required to predict the response of ozone levels to changes in emissions under the influence of climate change with confidence.
As discussed in the introduction, the non-linear relationship of ozone with ¸\ce{NO_x} and VOC levels makes this task even more difficult.
Thus work such as the studies performed as part of this thesis are needed to improve the confidence of such future predictions of ozone levels.

The detailed process studies performed in this study are a part of the process of seeking improvements to current AQ modelling were all performed using a box model.
While a box model is ideal for the scope of the studies and allow an insight into which chemical processes need to be focused on using more realistic 3-D models, the results here cannot be just scaled up to the real-world.
Further work using 3-D models are required to verify how processes, such as regional transport, included in 3-D models but not the box model, also influence ozone production.

A number of recommendations to the AQ modelling community can be listed from this work.
Firstly, to keep up to date with the recommendations from the chemical kineticy community, AQ modelling groups should use more recent versions of chemical mechanism.
Older chemical mechanisms such as RADM2 are still used by many modelling groups despite many updates in the chemistry. 
Also, the first study showed that updates to the RADM2 chemistry using the same underlying simplification techniques in the RACM and RACM2 chemcial mechanisms improve the representation of the secondary VOC degradation chemistry leading to more similar ozone production from VOCs to the more detailed MCM chemical mechanism.
This is undoubtably more work for a modelling group: testing new model with the new chemical mechanism and translating emissions into the new chemical species, for example.
Having said that, the MCM should not be assumed to be an exact representation of the tropospheric chemistry as many assumptions and simplifications were used when developing the MCM.
For example, the degradation of aromatic VOC has been shown to not reflect that of chamber experiments.
More work is needed on a laboratory scale to verify the assumptions used by the MCM and other chemical mechanisms.

The main source of the generally lower ozone production from VOC in reduced chemical mechanisms is the faster break down of the emitted VOC into smaller compounds.
This feature of reduced chemical mechanisms should be addressed when developing future chemical mechanisms, this might be at the expense of computational efficiency as more species and reactions would most likely need to be included.
Although gains in computational speed with modern computing centres might offset this.

Secondly, the results of the investigation into the effects of different speciations of the solvent sector showed that the effects on ozone production were noticable.
This initial scoping study looked only at maximum ozone production within the box model but given these results further work using more realistic models and conditions are required.
Also the speciations of NMVOCs by the emission inventories vary and this is also an area needing further work since different VOCs have different ozone production potentials, this was also shown in the second study where higher speciations of alkanes led to higher ozone production than emission inventories with higher speciations of oxygenated compounds.
Future emission reduction scenarios may look at substituting the emissions of more reactive NMVOC with less reactive NMVOC changing the ozone production potential of the emission inventory.
Depending on the type of emissions, this may reduce ozone production close to emission sources but increase ozone production downwind where the degradation of alkanes and oxygentated compounds influcenes.
These are all important questions to be addressed before developing new emission inventories that represent the speciations of NMVOC emissions from urban areas is a huge task requiring a combined effort of the emissions, ambient measurent and the modelling community to better represent VOC emissions from urban areas within models.

The future climate is predicted to be warmer as a result of climate change and this may affect the predicted ozone production chemistry from future emission scenarios.
The final study of this work considered only the effect of temperature, known to the major meteorological driver of ozone productio in many regions.
The influence of other meterological parameters also plays a role, as mentioned in the introduction.
Modelling studies looking at not only the effects of the individual meteorological parameters but also coupling of these parameters is also important due to competing effects on ozone production.

The final study concluded that the oxidation of emitted NMVOC strongly influenced ozone production with temperature which neatly shows the potential impacts of having representation of the initial NMVOCs from an emission inventory and representing the degradation chemistry by the chemical mechanism.
Since the ozone production with temperature was sensitive to mixing, this also shows the importance of representing the secondary degradation chemistry adequately by the chemical mechanism as if the chemistry is not properly represented then model predictions would underestimate ozone production with temperature.
The degradation of NMVOC should be slowed down in chemical mechanisms to better simulate ozone production from emissions downwind and that could influence levels of background ozone.

Having a limit value for ozone pollution in the EU would provide an incentive to address many of these issues and allow the modelling community to further improve the modelling of this tricky pollutant.
