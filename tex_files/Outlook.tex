Although ozone pollution is a major problem in Europe, there is currently no legally binding limit value for ozone in Europe.
The EU Directive 2008/50/EG sets a target value for human health requiring the mean eight hourly ozone concentration not to exceed $120$~$\mu$g~m$^{-3}$ (= $60$~ppbv) on more than $25$ calendar days.
The same EU Directive also sets an AOT40 target value for the exposure of vegetation to ozone of less than $18,000$~$\mu$g~m$^{-3} \cdot$~h, where AOT40 is the sum of the differences between the mean hourly ozone values above $80$~$\mu$g~m$^{-3}$ (= $40$~ppbv) and the value of $80$~$\mu$g~m$^{-3}$ between 8~am and 8~pm from May to July.

The EU does have laws regulating the emissions of the precursors (\ce{NO_x} and VOC) of ozone aiming to reduce the levels of ambient ozone.
These laws have indeed reduced the emissions of both \ce{NO_x} and VOC over Europe but despite these reductions in emissions the target value for ozone is regularly exceeded throughout Europe.
The non-linear relationship between ozone, \ce{NO_x} and VOC as well as increased intercontinental transport of ozone and its precursors impact the response of ozone pollution despite reductions in emissions of precursors.
Thus the European Environmental Agency (EEA) recommends further mitigation efforts on different scales: local, regional and global.

Setting a legally binding limit value for ambient ozone pollution by the EU should inspire further mitigation strategies at the local and European levels in a bid to meet this limit.
Meeting this limit would no doubt be a difficult task and requires planning of different mitigation approaches, AQ modelling is an important tool that can be used to determine how effective a mitigation strategy should be for reducing ambient ozone.
Thus research that aims to improve model performance so that more confidence can be had for the AQ predictions from models is particular important for the goal of meeting a limit value for ozone pollution.
The detailed process studies performed as part of this work were designed to ultimitely improve model performance and have more confidence in the predictions of AQ models of mitigation strategies.

Based on these detailed process studies, a number of recommendations to the AQ modelling community are listed.
AQ modelling groups should use up-to-date chemical mechanism to incorporate the findings and recommendations from the chemical kinetics community.
This is undoubtably more work for a modelling group as further work such as testing the model with a new chemical mechanism and translating emissions into the new chemical species need to be performed.
However as shown in the first study, updated versions of the same chemical mechanism produced more similar amounts of ozone to the near-explicit MCM~v3.2 chemical mechanism.
Thus using an updated chemical mechanism should increase the confidence of the modelled ozone production from the degradation of emitted VOCs.

As the lumped-intermediate chemical mechanism produced the most similar amounts of ozone to the MCM~v3.2, the approach of using a highly detailed chemical mechanism and lumping the degradation products appears promising for developing future chemical mechanisms.
This approach did not break down the emitted VOC into smaller degradation products as fast as the lumped-molecule and lumped-structure chemical mechanisms, which was the main cause for the lower ozone production compared to the MCM~v3.2.
Lumped-intermediate chemical mechanisms include more chemical species than lumped-molecule and lumped-structure chemical mechanisms making their use less appealling from a computational efficiency perspective.
However gains in computational speed with modern computing centres might reduce this concern and facilitate using more complex chemical mechanisms in regional modelling studies.

One feature of future mitigation strategies could be to replace the emissions of a more-reactive NMVOC with a less-reactive NMVOC thus changing the NMVOC speciation profile from emission sectors.
Such mitigation strategies would require updating emission inventories and an assessment on how the change in speciation would influence ambient ozone.
The second study showed that ozone production close to emission sources would be reduced using such mitigation strategies but ozone production downwind may increase.
This study was a first step into assessing the capabilities of a simple box model and more research is required to assess the impacts of emission substitution strategies on the regional scale.

The climate is predicted to be warmer in the future as a result of climate change and this may affect the predicted ozone production chemistry from future emission scenarios.
The influence of meteorological variables on ozone production is extremely important and requires a deeper understanding to ensure that mitigation strategies are robust enough to still reduce ambient ozone in the future.
The oxidation of emitted NMVOC determined the increase of ozone with temperature further emphasising the importance of representing both the speciation and degradation of emitted NMVOC.
Moreover, the increase of ozone with temperature was sensitive to atmospheric mixing.
With less atmospheric mixing, NMVOC degradation can produce more ozone than with higher levels of mixing stressing the importance of representing the secondary degradation of NMVOC by the chemical mechanism used by an AQ model.

The results from the detailed process studies performed in this work were performed using a box model and further work using 3D models are required to verify how processes, such as regional transport, also influence ozone production with these conditions.
The use of a box model was ideal for the scope of the studies of this thesis allowing a deeper insight into which chemical processes require more focus when using more realistic 3-D models.
