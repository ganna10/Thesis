Although ozone pollution is a major problem in Europe, there is currently no legally binding limit value for ozone.
The EU Directive 2008/50/EG sets a target value for human health requiring the mean eight hourly ozone concentration not to exceed $120$~$\mu$g~m$^{-3}$ (= $60$~ppbv) on more than $25$ calendar days.
The same EU Directive also sets an AOT40 target value for the exposure of vegetation to ozone of less than $18,000$~$\mu$g~m$^{-3} \cdot$~h. 
AOT40 is the sum of the differences between the mean hourly ozone values above $80$~$\mu$g~m$^{-3}$ (= $40$~ppbv) and the value of $80$~$\mu$g~m$^{-3}$ between 8~am and 8~pm from May to July.

The EU has laws regulating the emissions of ozone precursors (\ce{NO_x} and VOC).
While these laws have reduced the emissions of both \ce{NO_x} and VOC over Europe, the target value for ozone is regularly exceeded throughout Europe.
The non-linear relationship between ozone, \ce{NO_x} and VOC as well as increased intercontinental transport of ozone and its precursors impact the response of ozone pollution despite reductions in precursor emissions.
The European Environmental Agency (EEA) recommends further mitigation efforts on the local, regional and global scales.

Setting a legally binding limit value for ambient ozone pollution over Europe should inspire mitigation strategies at the local and regional scales in a bid to meet this limit.
Meeting a limit value requires assessing different mitigation approaches and here AQ modelling will be a vital tool in determining the efficacy of a mitigation strategy for reducing ambient ozone. 
Thus research aiming to improve model performance would also aid in increasing the confidence of AQ predictions from models, an asset for meeting a limit value for ozone pollution.
The detailed process studies performed as part of this work were designed to ultimately improve model performance increasing the confidence of the predictions of AQ models for mitigation strategies.

Based on these detailed process studies, a number of recommendations to the AQ modelling community are listed.
AQ modelling groups should use up-to-date chemical mechanisms to incorporate the findings and recommendations from the chemical kinetics community.
This is undoubtably more work for a modelling group as further work such as testing the model with a new chemical mechanism and translating emissions into the new chemical species would need to be performed.
However as shown in the first study, updated versions of the same chemical mechanism produced more similar amounts of ozone to the near-explicit MCM~v3.2 chemical mechanism.
Thus using an updated chemical mechanism should increase the confidence of the modelled ozone production from the degradation of emitted VOCs.

As the lumped-intermediate chemical mechanism produced the most similar amounts of ozone to the MCM~v3.2, the approach of using a highly detailed chemical mechanism and lumping the degradation products appears promising for developing future chemical mechanisms.
This approach did not break down the emitted VOC into smaller degradation products as fast as the lumped-molecule and lumped-structure chemical mechanisms, which was the main cause for the lower ozone production using these chemical mechanisms compared to the MCM~v3.2.
Lumped-intermediate chemical mechanisms include more chemical species than lumped-molecule and lumped-structure chemical mechanisms making their use less appealling from a computational efficiency perspective.
However gains in computational speed with modern computing centres might reduce this concern and facilitate the use of more complex chemical mechanisms as part of 3D models.

One feature of future mitigation strategies could be to substitute the emissions of a more-reactive NMVOC with a less-reactive NMVOC thus changing the NMVOC speciation profile from emission sectors.
Such mitigation strategies require updating emission inventories and assessing how the change in speciation could influence ambient ozone levels.
The results of the second study indicate that ozone production close to emission sources would be reduced using such mitigation strategies but ozone production downwind may increase.

A warmer climate is predicted in the future as a result of climate change and this may affect ozone production chemistry in future emission scenarios.
The influence of meteorological variables on ozone production is extremely important with the third study demonstrating an increase in ozone production with temperature.
A deeper understanding of the effects of meteorology on ozone production is required to ensure that mitigation strategies are robust enough to still reduce ambient ozone in the future.

The oxidation of emitted NMVOC determined the increase of ozone with temperature further emphasising the importance of adequately representing both the speciation and initial degradation of emitted NMVOC.
The increase of ozone with temperature was sensitive to atmospheric mixing with less atmospheric mixing allowing the secondary degradation of NMVOC to proceed further than situations with enhanced atmospheric mixing.
Thus stressing the importance of representing the secondary degradation of NMVOC by the chemical mechanism used by an AQ model.

The results from the detailed process studies performed in this work were all performed using a box model and further work using 3D models is required to verify how additional processes, such as regional transport, also influence ozone production with these conditions.
The use of a box model was ideal for the scope of the studies of this thesis allowing a deeper insight into the chemical processes requiring a sharper focus when using more realistic 3-D models.
