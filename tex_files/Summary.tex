\section{Summary}
Tropospheric ozone is a short-lived climate forcing pollutant that is hazardous to human health and impacts deleteriously on vegetation.
Ozone is not emitted directly into the troposphere but formed from the photochemical reactions of VOCs and \ce{NO_x} with meteorological conditions strongly influencing ozone production.
This thesis assessed the chemical mechanisms of ozone production chemistry represented within air quality models by determining the influence of VOC degradation, the speciation of VOC emissions and temperature on modelled ozone production.
In order to focus on the impacts of the representation of tropospheric chemistry on ozone production, all modelling experiments in this work used a box model.
The highly-detailed MCM~v3.2 chemical mechanism was used as a reference and the model simulations were then repeated using reduced chemical mechanisms typically used by regional and global models.

The effects of different simplification approaches used by chemical mechanisms on ozone production were determined by comparing the ozone produced during VOC degradation between different chemical mechanisms.
The lumped-intermediate (CRI~v2) chemical mechanism produced the most similar amounts of ozone to the MCM~v3.2 from the degradation of each emitted VOC.
On the other hand, VOC degradation described by lumped-molecule (MOZART-4, RADM2, RACM and RACM2) and lumped-structure (CBM-IV and CB05) chemical mechanisms generally produced less ozone than the MCM~v3.2.
The lower ozone production from VOC degradation in the lumped-molecule and lumped-structure chemical mechanisms was caused by a faster breakdown of the emitted VOC into smaller sized degradation products.

The influence of the speciation of VOC emissions by an emission inventory on ozone production was established by comparing the ozone produced from different emission inventories of the solvent sector.
In these experiments, both the different emission inventories of VOC emissions and chemical mechanisms (MCM~v3.2, MOZART-4, RADM2) led to differences in peak ozone mixing ratios and ozone production.
These results indicated a sensitivity of ozone production to both the choice of chemical mechanism and emission inventory speciation.
The ozone production on the first day was influenced by the specified contributions of alkene and aromatic VOC.
Emission inventories specifying larger amounts of alkane emissions produced the largest amounts of ozone at the end of simulations than those specifying a larger contribution of emissions of oxygenated VOC.

The final study of this work considered the relationship between ozone, temperature and \ce{NO_x}.
The increase of ozone with temperature due to temperature-dependent chemistry was slightly larger than the increase of ozone with temperature due to temperature-dependent isoprene emissions in all \ce{NO_x}-conditions.
A non-linear relationship between ozone, temperature and \ce{NO_x} was obtained with each chemical mechanism used in this study (MCM~v3.2, CRI~v2, MOZART-4, RADM2 and CB05).
With each chemical mechanism, the temperature-dependent chemistry of VOC-oxidation and peroxy nitrates, such as PANs, caused the increase in ozone production with temperature.

The detailed process studies of this work will help improve model performance and the confidence in predictions of future ozone levels from air quality modelling studies.
The representation of VOC degradation and secondary processes, such as the rate of breakdown of the emitted VOCs and peroxy nitrate chemistry by a chemical mechanism, are particularly important processes for the predictions of ozone levels with different mitigation strategies.

%\newpage
\section{Zusammenfassung}
Troposphärisches Ozon ist ein kurzlebiger klimawirksamer Schadstoff, der für die menschliche Gesundheit gefährlich ist und sich auf schädliche Weise auf die Vegetation auswirkt.
Ozon wird nicht direkt in die Troposphäre emittiert, sondern entsteht aus den photochemischen Reaktionen von flüchtigen organischen Verbindungen (VOCs) und \ce{NO_x}, wobei meteorologische Bedingungen die Ozonproduktion stark beeinflussen.
%Die vorliegende Arbeit bewertet die Chemiemechanismen zur Bechrechnung der Ozonproduktion Atmosphären-Chemie-Modellen, indem der Einfluss des VOC-Abbaus, des Anteils verschiedener VOC-Komponenten an den VOC-Gesamtemissionen (\quotedblbase VOC Speciation``) und der Temperatur auf modellierte Ozonproduktion bestimmen wurde.
In der vorliegenden Arbeit wurden verschiedene Chemiemechanismen zur Berechnung der Ozonproduktion in Atmosphären-Chemie-Modellen in Bezug auf den Einfluß des VOC-Abbaus, des Anteils verschiedener VOC-Komponenten an den VOC-Gesamtemissionen (\quotedblbase VOC Speciation``) und der Temperatur auf die modellierte Ozonproduktion untersucht.
%Alle Modellexperimente in dieser Arbeit verwendeten ein Box-Modell, um detaillierte Prozessstudien mit dem Fokus auf der Darstellung der Auswirkung troposphärischer Chemie auf die Ozonproduktion durchzuführen.
Um detaillierte Prozessstudien mit einem Fokus auf die Repräsentativität der Auswirkung troposphärischer Chemie auf die Ozonproduktion in den Chemiemechanismen durchzuführen, wurden in dieser Arbeit Modellexperimente mit einem Box-Modell durchgeführt.
Als Referenz wurde zunächst der sehr detaillierte Chemiemechanismus MCM~v3.2 in Modellsimulationen verwendet.
Diese Simulationen wurden dann unter der Verwendung von reduzierten Chemiemechanismen, welche typischerweise von regionalen und globalen Modellen verwendet werden, wiederholt. 

Um die Auswirkungen der verschiedenen Vereinfachungsansätze, die in den Chemiemechanismen bezüglich der Ozonproduktion verwenden werden, auf die Ozonproduktion zu bestimmen, wurde die durch die verschieden Chemiemechanismen simulierte Ozonproduktion während des VOC-Abbaus verglichen.
Der Chemiemechanismus CRI~v2 mit zusammengefassten Zwischenprodukten (\quotedblbase lumped-intermediate``) hat im Vergleich zu  MCM~v3.2 die vergleichbarste Ozonmenge aus dem Abbau jedes emittierte VOC erzeugt. 
Der VOC-Abbau, welcher in den Chemiemechanismen mit zusammengefassten Molekülen (\quotedblbase lumped-molecule``) (MOZART-4, RADM2, RACM and RACM2) und zusammengefassten Strukturen (\quotedblbase lumped-structure``) (CBM-IV and CB05) beschrieben ist, produzierte jedoch gemeinhin weniger Ozon als MCM~v3.2. 
Die geringere Ozonproduktion während des VOC-Abbaus in den \quotedblbase lumped-molecule`` und \quotedblbase lumped-structure`` Chemiemechanismen wurde durch ein schnellerer Zerfall des emittierten VOC in kleinere Abbauprodukte verursacht.

Der Einfluss der \quotedblbase VOC Speciation`` von VOC-Emissionen durch ein Emissionsinventar wurde durch einen Vergleich des aus verschiedenen Emissionsinventaren des Lösungsmittelsektors produzierten Ozons festgestellt.
In diesen Experimenten führte die Verwendung unterschiedlicher VOC-Emissionsinventare und Chemiemechanismen (MCM~v3.2, MOZART-4, RADM2) zu Unterschieden bei den Spitzenwerten der Ozonmischungsverhältnisse sowie der Ozonproduktion.
Diese Ergebnisse zeigten eine Abhängigkeit der Ozonproduktion sowohl von der Wahl des Chemiemechanismus als auch von den \quotedblbase VOC Speciation`` in den Emissionsinventaren.
Am ersten Tag wurde die Ozonproduktion von den spezifizierten Beiträgen von Alken und aromatischen VOC beeinflusst.
Die Emissionsinventare mit höheren Mengen an Alkanemissionen erzeugten die größten Mengen an Ozon am Ende der Simulation, im Vergleich zu denen, die einen größeren Beitrag des Ausstoßes von oxygeniertem VOC definierten.
%Im Vergleich zu den Emissionsinventaren mit einem höheren festgelegten Beitrag con oxygeniertem VOC, wird eine höhere Gesamtmenge an Ozon unter der Verwendung von Emissionsinventaren mit einem höheren Beitrag von Alkanemissionen erzeugt.

Die letzte Studie dieser Arbeit betrachtete die Beziehung zwischen Ozon, Temperatur und \ce{NO_x}.
Der Anstieg des Ozons mit der Temperatur aufgrund der temperaturabhängigen Chemie war geringfügig größer als der Anstieg des Ozons mit der Temperatur aufgrund der temperaturabhängigen Isoprenemissionen unter allen \ce{NO_x}-Bedingungen.
Ein nicht-linearer Zusammenhang zwischen Ozon, Temperatur und \ce{NO_x} wurde mit jedem in dieser Studie verwendeten Chemiemechanismus (MCM~v3.2, CRI~v2, MOZART-4, RADM2 und CB05) aufgezeigt.
Bei jedem Chemiemechanismus kam es aufgrund der temperaturabhängigen Chemie von VOC-Abbau und Peroxynitraten, wie z.~B. PANs, zu einem Anstieg der Ozonproduktion mit der Temperatur. 

Die detaillierten Prozessstudien dieser Arbeit werden dazu beitragen, die Güte von Modellen und das Vertrauen in die Prognosen zukünftiger Ozonwerte von Studien mit Atmosphären-Chemie-Modellen zu verbessern.
Die Darstellung des VOC-Abbaus und sekundärer Prozesse, wie der Zerfallgeschwindigkeit von emittierten VOCs sowie der Peroxynitratchemie durch einen Chemiemechanismus, sind besonders wichtige Prozesse für die Vorhersage der Ozonwerte unter Anwendung von unterschiedlichen Minderungsstrategien.
