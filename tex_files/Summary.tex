\section{Summary}
Tropospheric ozone is a short-lived climate pollutant that is harmful to both the human population and impacts deleteriously on vegetation.
Ozone is not emitted directly into the troposphere but formed from the photochemical reactions of VOCs and \ce{NO_x} with meteorology strongly influencing ozone production.
This thesis assessed the detailed chemical mechanisms of ozone production represented within AQ models by determining the influences of VOC degradation. the speciation of VOC emissions and temperature on modelled ozone production.
All modelling experiments in this work used a box model to perform detailed process studies focussing on the representation of tropospheric chemistry related to ozone production.
Model simulations were set up using the highly-detailed MCM~v3.2 chemical mechanism and then repeating using reduced chemical mechanisms typically used by regional and global models to determine the sensitivity of ozone production under different conditions to the choice of chemical mechanism.
Thus in all experiments the MCM~v3.2 was the reference chemical mechanism.

The effects of different simplification approaches to ozone production chemistry used by models on ozone production were determined.
The lumped-intermediate (CRI~v2) chemical mechanism produced the most similar amounts of ozone to the MCM~v3.2 from the degradation of each VOC considered in the study.
Lumped-molecule (MOZART-4, RADM2, RACM and RACM2) and lumped-structure (CBM-IV and CB05) chemical mechanisms generally produced less ozone than the MCM~v3.2 from each VOC.
The faster breakdown of the emitted VOC into smaller sized degradation products in the lumped-molecule and lumped-structure chemical mechanisms caused the lower ozone production from the VOC.
Also, larger differences in ozone production were produced from VOC represented by mechanism species rather than being represented by explicit species indicating a sensitivity of ozone production to the representation of VOC emissions by the chemical mechanism.

The influence of the speciation of VOC emissions on ozone production were determined by comparing the ozone produced from different speciations of the solvent sector.
In these experiments, the different speciations of VOC emissions led to differences in peak ozone and also cumulative ozone production.
Solvent sector speciations that specified larger amounts of alkane emissions produced the largest cumulative ozone production at the end of simulations than solvent sector speciations that specified larger emissions of oxygenated VOC.
The ozone production on the first day was influenced by the specified amounts of alkene and aromatic VOC.
Repeating experiments with reduced chemical mechanisms reproduced the differences in ozone production obtained when using the MCM~v3.2 chemical mechanism.
These results indicate a sensitivity of ozone production to both the choice of chemical mechanism and emission inventory speciation.

The final study of this work considered the relationship between ozone, temperature and \ce{NO_x}.
The increase of ozone with temperature due to temperature-dependent chemistry was slightly more than the increase of ozone with temperature due to increased isoprene emissions with temperature.
A non-linear relationship between ozone, temperature and \ce{NO_x} was obtained with each chemical mechanism used in this study (MCM~v3.2, CRI~v2, MOZART-4, RADM2 and CB05).
With each chemical mechanism, the increase in ozone production with temperature was due to the increased loss of the emitted VOC with temperature mainly caused by the increased OH-reactivity of each VOC with temperature.
The temperature-dependent chemistry of peroxy nitrate compounds, such as PANs, also led to an increase of ozone production with temperature.

Determining the importance of ozone production on VOC degradation chemistry, the speciation of VOC emissions and temperature and the representation of these processes by the chemical mechanism used by AQ models will help in improving model performance and the confidence in ozone predictions from modelling studies.
The representation of VOC degradation and secondary processes such as the rate of breakdown of the emitted VOC and peroxy nitrate chemistry are all important processes that will aid in predicting ozone pollution under different mitigation scenarios.
Furthermore, if a legally-binding limit value for ozone pollution is implemented in Europe the importance of increased confidence in modelling predictions will be magnified.

\section{Zusammenfassung}
