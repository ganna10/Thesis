\section{Summary}
Tropospheric ozone is a short-lived climate forcing pollutant that is hazardous to human health and impacts deleteriously on vegetation.
Ozone is not emitted directly into the troposphere but formed from the photochemical reactions of VOCs and \ce{NO_x} with meteorological conditions strongly influencing ozone production.
This thesis assessed the chemical mechanisms of ozone production represented within air quality models by determining the influence of VOC degradation, the speciation of VOC emissions and temperature on modelled ozone production.
All modelling experiments in this work used a box model to perform detailed process studies focusing on the representation of tropospheric chemistry impacting ozone production.
Model simulations used the highly-detailed MCM~v3.2 chemical mechanism as a reference and repeated using reduced chemical mechanisms typically used by regional and global models determining the sensitivity of ozone production to the choice of chemical mechanism.

The effects of different simplification approaches used by chemical mechanisms on ozone production were determined by comparing the ozone produced during VOC degradation between different chemical mechanisms.
The lumped-intermediate (CRI~v2) chemical mechanism produced the most similar amounts of ozone to the MCM~v3.2 from the degradation of each VOC.
On the other hand, VOC degradation described by lumped-molecule (MOZART-4, RADM2, RACM and RACM2) and lumped-structure (CBM-IV and CB05) chemical mechanisms generally produced less ozone than the MCM~v3.2.
A faster breakdown of the emitted VOC into smaller sized degradation products in the lumped-molecule and lumped-structure chemical mechanisms caused the lower ozone production during VOC degradation.
Also, larger differences in ozone production were produced from VOC represented by mechanism species than VOCs represented by explicit species.

The influence on ozone production from the speciation of VOC emissions by an emission inventory was established by comparing the ozone produced from different emission inventories of the solvent sector.
In these experiments, the different emission inventories of VOC emissions led to differences in peak ozone mixing ratios and ozone production.
The ozone production on the first day was influenced by the specified contributions of alkene and aromatic VOC.
Emission inventories specifying larger amounts of alkane emissions produced the largest amounts of ozone at the end of simulations than those specifying a larger contribution of emissions of oxygenated VOC.
Repeating the simulations with reduced chemical mechanisms (MOZART-4 and RADM2) reproduced the differences in ozone production obtained with the MCM~v3.2 chemical mechanism.
These results indicated a sensitivity of ozone production to both the choice of chemical mechanism and emission inventory speciation.

The final study of this work considered the relationship between ozone, temperature and \ce{NO_x}.
The increase of ozone with temperature due to temperature-dependent chemistry was slightly larger than the increase of ozone with temperature due to increased isoprene emissions with temperature in all \ce{NO_x}-conditions.
A non-linear relationship between ozone, temperature and \ce{NO_x} was obtained with each chemical mechanism used in this study (MCM~v3.2, CRI~v2, MOZART-4, RADM2 and CB05).
With each chemical mechanism, the increase in ozone production with temperature was due to the increased loss of the emitted VOCs with temperature, mainly caused by the increased OH-reactivity of VOC with temperature.
The temperature-dependent chemistry of peroxy nitrate compounds, such as PANs, also led to an increase of ozone production with temperature.

The detailed processes studies of this work will help improve model performance and the confidence in predictions of future ozone levels from air quality modelling studies.
The representation of VOC degradation and secondary processes, such as the rate of breakdown of the emitted VOCs and peroxy nitrate chemistry by a chemical mechanism, are particularly important processes for the predictions of ozone levels with different mitigation strategies.

\newpage
\section{Zusammenfassung}
Troposphärisches Ozon ist ein kurzlebiger klimawirksamer Schadstoff, der für die menschliche Gesundheit gefährlich ist und sich auf schädliche Weise auf die Vegetation auswirkt.
Ozon wird nicht direkt in die Troposphäre ausgestoßen, sondern entsteht aus den photochemischen Reaktionen von VOCs und \ce{NOx}, wobei meteorologische Bedingungen die Ozonproduktion stark beeinflussen.
Die vorliegende Arbeit untersuchte die Chemiemechanismen der Ozonproduktion innerhalb von Luftqualitätsmodellen, indem der Einfluss von VOC-Abbau, des Anteils verschiedener VOC-Komponenten am Gesamt-VOC-Ausstoß (\quotedblbase VOC Speciation``) und der Temperatur auf modellierte Ozonproduktion untersucht wurde.
Alle Modellierungsexperimente in dieser Arbeit verwendeten ein Box-Modell, um detaillierte Prozessstudien mit dem Fokus auf der Darstellung der Auswirkung troposphärischer Chemie auf die Ozonproduktion durchzuführen.
Modellsimulationen verwendeten den hochdetaillierten Chemiemechanismus MCM~v3.2 als Referenz und wurden dann mit reduzierten Chemiemechanismen, die typischerweise von regionalen and globalen Modellen verwendet werden, wiederholt, um zu bestimmen, wie sich die Wahl von Chemiemechanismen auf die Empfindlichkeit der Ozonproduktion auswirkt. 

Die Auswirkungen der verschiedenen Vereinfachungsansätze, die die Chemiemechanismen bezüglich der Ozonproduktion verwenden, wurden bestimmt indem die durch die Chemiemechanismen simulierte Ozonproduktion während des VOC-Abbaus verglichen wurde.
Der Chemiemechanismus CRI~v2 mit zusammengefassten Zwischenprodukten hat im Vergleich zu  MCM~v3.2 die vergleichbarste Ozonmenge aus dem Abbau jedes VOC erzeugt. 
Der VOC-Abbau, welcher in den Chemiemechanismen mit zusammengefassten Molekülen (MOZART-4, RADM2, RACM and RACM2) und zusammengefassten Strukturen (CBM-IV and CB05) beschrieben ist, produzierte jedoch gemeinhin weniger Ozon als MCM~v3.2. 
Ein schnellerer Zerfall des ausgestoßenen VOC in kleinere Abbauprodukte in den Chemiemechanismen mit zusammengefassten Molekülen und zusammengefassten Strukturen verursachte die geringere Ozonproduktion während des VOC-Abbaus. 
Des Weiteren wurden größere Unterschiede in der Ozonproduktion durch VOCs erzeugt, welche durch Mechanismusspezien repräsentiert waren, als solche die durch explizite Spezien dargestellt wurden.

Der Einfluss auf die Ozonproduktion der \quotedblbase VOC Speciation`` von VOC-Ausstöße durch eine Emissionsinventarliste wurde durch einen Vergleich des aus verschiedenen Emissionsinventarlisten des Lösungsmittelsektors produzierten Ozons festgestellt.
In diesen Experimenten führten die unterschiedlichen Emissionsinventarlisten der VOC-Ausstöße zu Unterschieden bei den Spitzenwerten der Ozonmischungsverhältnisse sowie der Ozonproduktion.
Die Ozonproduktion am ersten Tag wurde von den angegebenen Beiträgen von Alken und aromatischen VOC beeinflusst.
Die Emissionsinventarlisten mit größeren Mengen an Alkanausstößen erzeugten die größten Mengen an Ozon am Ende der Simulation, im Vergleich zu denen, die einen größeren Beitrag des Ausstoßes von oxygeniertem VOC definierten.
Eine Wiederholung der Simulationen mit reduzierten Chemiemechanismen (MOZART-4 und RADM2) reproduzierten die Unterschiede in der Ozonproduktion, die auch mit dem Chemiemechanismus MCM~v3.2 erlangt wurden.
Diese Ergebnisse zeigten eine Abhängigkeit der Ozonproduktion sowohl von der Wahl des Chemiemechanismus als auch von den \quotedblbase VOC Speciaztion``-Emissionsinventarlisten.

Die letzte Studie dieser Arbeit betrachtete die Beziehung zwischen Ozon, Temperatur und \ce{NO_x}.
Der Anstieg des Ozons mit der Temperatur aufgrund der temperaturabhängigen Chemie war geringfügig größer als der Anstieg des Ozons mit der Temperatur aufgrund der erhöhten Isopren-Ausstöße mit der Temperatur in allen \ce{NO_x}-Bedingungen.
Ein nicht-linearer Zusammenhang zwischen Ozon, Temperatur und \ce{NO_x} wurde mit jedem in dieser Studie verwendeten Chemiemechanismus (MCM~v3.2, CRI~v2, MOZART-4, RADM2 und CB05) aufgezeigt.
Bei jedem Chemiemechanismus kam es aufgrund des erhöhten Verlusts des ausgestoßenen VOC in Zusammenhang mit der Temperatur zu einem Anstieg der Ozonproduktion mit der Temperatur. Diese wurde hauptsächlich durch die erhöhte OH-Reaktivität von VOC mit der Temperatur verursacht.
Die temperaturabhängige Chemie von Peroxynitraten, wie z.B. PANs, führte ebenfalls zu einer Erhöhung der Ozonproduktion mit der Temperatur.

Die detaillierten Prozessstudien dieser Arbeit werden dazu beitragen, die Leistungsfähigkeit von Modellen und das Vertrauen in die Prognosen zukünftiger Ozonwerte von Luftqualitätsmodellierungsstudien zu verbessern.
Die Darstellung des VOC-Abbaus und sekundärer Prozesse, wie die Zerfallgeschwindigkeit der ausgestoßenen VOCs und Peroxynitratchemie durch einen Chemiemechanismus, sind besonders wichtige Prozesse für die Vorhersage der Ozonwerte mit unterschiedlichen Minderungsstrategien.
