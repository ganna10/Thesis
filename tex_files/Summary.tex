\section{Summary}
Tropospheric ozone is a short-lived climate forcing pollutant that is hazardous to human health and impacts deleteriously on vegetation.
Ozone is not emitted directly into the troposphere but formed from the photochemical reactions of VOCs and \ce{NO_x} with meteorological conditions strongly influencing ozone production.
This thesis assessed the chemical mechanisms of ozone production represented within air quality models by determining the influence of VOC degradation, the speciation of VOC emissions and temperature on modelled ozone production.
All modelling experiments in this work used a box model to perform detailed process studies focusing on the representation of tropospheric chemistry impacting ozone production.
Model simulations used the highly-detailed MCM~v3.2 chemical mechanism as a reference and repeated using reduced chemical mechanisms typically used by regional and global models determining the sensitivity of ozone production to the choice of chemical mechanism.

The effects of different simplification approaches used by chemical mechanisms on ozone production were determined by comparing the ozone produced during VOC degradation between different chemical mechanisms.
The lumped-intermediate (CRI~v2) chemical mechanism produced the most similar amounts of ozone to the MCM~v3.2 from the degradation of each VOC.
While VOC degradation described by lumped-molecule (MOZART-4, RADM2, RACM and RACM2) and lumped-structure (CBM-IV and CB05) chemical mechanisms generally produced less ozone than the MCM~v3.2.
A faster breakdown of the emitted VOC into smaller sized degradation products in the lumped-molecule and lumped-structure chemical mechanisms caused the lower ozone production during VOC degradation.
Also, larger differences in ozone production were produced from VOC represented by mechanism species than VOCs represented by explicit species.

The influence on ozone production from the speciation of VOC emissions by an emission inventory was established by comparing the ozone produced from different emission inventories of the solvent sector.
In these experiments, the different emission inventories of VOC emissions led to differences in peak ozone mixing ratios and ozone production.
The ozone production on the first day was influenced by the specified contributions of alkene and aromatic VOC.
Emission inventories specifying larger amounts of alkane emissions produced the largest amounts of ozone at the end of simulations than those specifying a larger contribution of emissions of oxygenated VOC.
Repeating the simulations with reduced chemical mechanisms (MOZART-4 and RADM2) reproduced the differences in ozone production obtained with the MCM~v3.2 chemical mechanism.
These results indicated a sensitivity of ozone production to both the choice of chemical mechanism and emission inventory speciation.

The final study of this work considered the relationship between ozone, temperature and \ce{NO_x}.
The increase of ozone with temperature due to temperature-dependent chemistry was slightly larger than the increase of ozone with temperature due to increased isoprene emissions with temperature in all \ce{NO_x} conditions.
A non-linear relationship between ozone, temperature and \ce{NO_x} was obtained with each chemical mechanism used in this study (MCM~v3.2, CRI~v2, MOZART-4, RADM2 and CB05).
With each chemical mechanism, the increase in ozone production with temperature was due to the increased loss of the emitted VOC with temperature mainly caused by the increased OH-reactivity of VOC with temperature.
The temperature-dependent chemistry of peroxy nitrate compounds, such as PANs, also led to an increase of ozone production with temperature.

The detailed processes studies of this work will help improve model performance and the confidence in predictions of future ozone levels from air quality modelling studies.
The representation of VOC degradation and secondary processes such as the rate of breakdown of the emitted VOC and peroxy nitrate chemistry by a chemical mechanism are particularly important processes for the predictions of ozone levels with different mitigation strategies.
%Furthermore, if a legally-binding limit value for ozone pollution is implemented in Europe the importance of increased confidence in modelling predictions will be magnified.

\newpage
\section{Zusammenfassung}
Troposphärisches Ozon ist ein kurzlebiger klimawirksamer Schadstoff, der für die menschliche Gesundheit gefährlich ist und sich auf schädliche Weise auf die Vegetation auswirkt.
Ozon wird nicht direkt in die Troposphäre ausgestoßen, sondern entsteht aus den photochemischen Reaktionen von VOCs und \ce{NOx}, wobei meteorologische Bedingungen die Ozonproduktion stark beeinflussen.
Die vorliegende Arbeit untersuchte die chemischen Mechanismen der Ozonproduktion innerhalb von Luftqualitätsmodellen, indem der Einfluss von VOC-Abbau, der Speziation von VOC-Ausstößen und der Temperatur auf modellierte Ozonproduktion untersucht wurde.
Alle Modellierungsexperimente in dieser Arbeit verwendeten ein Box-Modell, um detaillierte Prozessstudien mit dem Fokus auf der Darstellung der Auswirkung troposphärischer Chemie auf die Ozonproduktion durchzuführen.
Modellsimulationen verwendeten den hochdetaillierten, chemischen Mechanismus MCM~v3.2 als Referenz. Die Modellsimulation wurden dann mit reduzierten chemischen Mechanismen wiederholt, die in der Regel von regionalen und globalen Modellen verwendet werden, um zu bestimmen wie die Wahl der chemischen Mechanismen sich auf die Empfindlichkeit der Ozonproduktion auswirkt.

Die Auswirkungen der verschiedenen Vereinfachungsansätze, die die chemischen Mechanismen bezüglich der Ozonproduktion verwenden, wurden durch einen Vergleich der Ozonproduktion während des VOC-Abbaus zwischen verschiedenen chemischen Mechanismen bestimmt.
Der chemische Mechanismus CRI~v2 hat im Vergleich zu  MCM~v3.2 die vergleichbarste Ozonmenge aus dem Abbau jedes VOC erzeugt. Der VOC-Abbau welcher in den xxx (M-4, R2, R und R2) und yyy (C-IV und CB) chemischen Mechanismen beschrieben ist, produzierte jedoch gemeinhin weniger Ozon als MCM~v3.2. 
Ein schnellerer Abbau des ausgestoßenen VOC in kleinere Abbauprodukte in den xxx und yyy chemischen Mechanismen verursachte die geringere Ozonproduktion während des VOC-Abbaus. 
Des Weiteren wurden größere Unterschiede in der Ozonproduktion durch VOCs erzeugt, welche durch Mechanismusspezien repräsentiert waren, als solche die durchexplizite Spezien dargestellt wurden.

---

Der Einfluss auf die Ozonproduktion von der Speziation von VOC-Ausstöße durch ein Emissionsinventar wurde durch Vergleich der Ozon aus verschiedenen Emissionsinventaren des Lösungsmittels Sektor produziert etabliert.
In diesen Experimenten führte die unterschiedlichen Emissionsinventare der VOC-Ausstöße von Unterschieden in der Mischungsverhältnisse von Spitzozon und die Ozonproduktion.
Die Ozonproduktion am ersten Tag wurde von den angegebenen Beiträgen von Alken und aromatischen VOC beeinflußt.
Emissionsinventar größere Mengen an Alkanausstöße Angabe erzeugt die größten Mengen an Ozon am Ende der Simulation als die einen größeren Beitrag der Ausstößen von oxygeniertem VOC angibt.
Durch Wiederholung der Simulationen mit reduzierten chemischen Mechanismen (MOZART-4 und RADM2) wiedergegeben, die Unterschiede in der Ozonproduktion mit dem MCM~v3.2 chemischen Mechanismus erhalten.
Diese Ergebnisse zeigten eine Sensitivität von Ozonproduktion sowohl die Wahl der chemischen Mechanismus und Emissionsinventar Speziation.

Die letzte Studie dieser Arbeit betrachtet die Beziehung zwischen Ozon, Temperatur und \ce{NO_x}.
Der Anstieg des Ozons mit der Temperatur aufgrund der temperaturabhängigen Chemie war geringfügig größer als der Anstieg des Ozons mit der Temperatur aufgrund der erhöhten Isopren-Ausstöße mit der Temperatur in allen \ce{NO_x} Bedingungen.
Eine nicht-lineare Zusammenhang zwischen Ozon, Temperatur und \ce{NO_x} wurde mit jeder chemischen Mechanismus in dieser Studie (MCM~v3.2, CRI~v2, MOZART-4, RADM2 und CB05) verwendet wird, erhalten.
Bei jeder chemischen Mechanismus, war der Anstieg der Ozonproduktion mit der Temperatur aufgrund der erhöhten Verlust des ausgestoßenen VOC mit der Temperatur hauptsächlich verursacht durch die erhöhte OH-Reaktivität von VOC mit der Temperatur.
Die temperaturabhängige Chemie von Peroxynitraten, wie PANs, führte auch zu einer Erhöhung der Ozonproduktion mit der Temperatur.

Die detaillierten Prozesse Studien dieser Arbeit wird dazu beitragen, Modellleistung und das Vertrauen in die Prognosen zukünftiger Ozonwerte von Luftqualitätsmodellierungsstudien zu verbessern.
Die Darstellung von VOC-Abbau und sekundären Prozesse wie die Abbaugeschwindigkeit der emittierten VOC und Peroxynitratchemie durch einen chemischen Mechanismus sind besonders wichtige Prozesse für den Vorhersagen der Ozonwerte mit unterschiedlichen Minderungsstrategien.
