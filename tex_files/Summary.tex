\section{Summary}
Tropospheric ozone is a short-lived climate pollutant that is hazardous to human health and impacts deleteriously on vegetation.
Ozone is not emitted directly into the troposphere but formed from the photochemical reactions of VOCs and \ce{NO_x} with meteorological conditions strongly influencing ozone production.
This thesis assessed the chemical mechanisms of ozone production represented within AQ~models by determining the influence of VOC degradation, the speciation of VOC emissions and temperature on modelled ozone production.
All modelling experiments in this work used a box model to perform detailed process studies focusing on the representation of tropospheric chemistry impacting ozone production.
Model simulations used the highly-detailed MCM~v3.2 chemical mechanism as a reference and were then repeating using reduced chemical mechanisms typically used by regional and global models determining the sensitivity of ozone production to the choice of chemical mechanism.

The effects of different simplification approaches used by chemical mechanisms on ozone production were determined by comparing the ozone produced during VOC degradation between different chemical mechanisms.
The lumped-intermediate (CRI~v2) chemical mechanism produced the most similar amounts of ozone to the MCM~v3.2 from the degradation of each VOC.
While VOC degradation described by lumped-molecule (MOZART-4, RADM2, RACM and RACM2) and lumped-structure (CBM-IV and CB05) chemical mechanisms generally produced less ozone than MCM~v3.2.
A faster breakdown of the emitted VOC into smaller sized degradation products in the lumped-molecule and lumped-structure chemical mechanisms caused the lower ozone production during VOC degradation.
Also, larger differences in ozone production were produced from VOC represented by mechanism species than VOCs represented by explicit species.

The influence on ozone production by the speciation of VOC emissions by an emission inventory was determined by comparing the ozone produced from different emission inventories of the solvent sector.
In these experiments, the different emission inventories of VOC emissions led to differences in peak ozone mixing ratios and ozone production.
The ozone production on the first day was influenced by the specified contributions of alkene and aromatic VOC.
Emission inventories specifying larger amounts of alkane emissions produced the largest amounts of ozone at the end of simulations than those specifying a larger contribution of emissions of oxygenated VOC.
Repeating the simulations with reduced chemical mechanisms (MOZART-4 and RADM2) reproduced the differences in ozone production obtained with the MCM~v3.2 chemical mechanism.
These results indicated a sensitivity of ozone production to both the choice of chemical mechanism and emission inventory speciation.

The final study of this work considered the relationship between ozone, temperature and \ce{NO_x}.
The increase of ozone with temperature due to temperature-dependent chemistry was slightly larger than the increase of ozone with temperature due to increased isoprene emissions with temperature in all \ce{NO_x} conditions.
A non-linear relationship between ozone, temperature and \ce{NO_x} was obtained with each chemical mechanism used in this study (MCM~v3.2, CRI~v2, MOZART-4, RADM2 and CB05).
With each chemical mechanism, the increase in ozone production with temperature was due to the increased loss of the emitted VOC with temperature mainly caused by the increased OH-reactivity of VOC with temperature.
The temperature-dependent chemistry of peroxy nitrate compounds, such as PANs, also led to an increase of ozone production with temperature.

The detailed processes studies of this work will help improve model performance and the confidence in predictions of future ozone levels from AQ~modelling studies.
The representation of VOC degradation and secondary processes such as the rate of breakdown of the emitted VOC and peroxy nitrate chemistry by a chemical mechanism are particularly important processes for the predictions of ozone levels with different mitigation strategies.
%Furthermore, if a legally-binding limit value for ozone pollution is implemented in Europe the importance of increased confidence in modelling predictions will be magnified.

\section{Zusammenfassung}
