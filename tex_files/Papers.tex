This chapter will outline the main findings in each of the scientific papers that were published as part of the PhD.

\section{Paper 1: A comparison of chemical mechanisms using tagged ozone production potential (TOPP) analysis} \label{s:chemical_mechanism_results}

Published: \bibentry{Coates:2015}.
\vspace{5mm}

The first paper described a box modelling study in which the secondary chemistry represented in many reduced chemical mechanisms (Table \todo{table}) for VOC typical of urban environments (Table 2 of the article) were compared to the detailed MCM chemical mechanisms.
The research question addressed in this paper was to verify whether these different representations of this secondary chemistry influence ozone production.

The degradation of each VOC prescribed in each chemical mechanism was ``tagged'' so that the \ce{O_x} production, a proxy for \ce{O3} production, could be attributed to the individual VOC sources.
Tagging the chemical mechanisms involved labelling every organic degradation product from a VOC with the name of the emitted VOC, thus each VOC has a separate set of reactions fully describing its degradation until the final products, \ce{CO2} and \ce{H2O}, are produced.

The ozone mixing ratios from reduced chemical mechanisms were generally lower than the mixing ratios from the reference MCM chemical mechanisms on the first two days of the simulations.
The VOC degradation prescribed in CRI~v2, a lumped-intermediate mechanism, produced the most similar amounts of \ce{O_x} to the MCM~v3.2 for each VOC.
Thus, the approach of using lumped-intermediate species whose degradation are based upon more detailed chemical mechanisms is preferable when developing future chemical mechanisms.

Many VOC are broken down into smaller-sized degradation products faster on the first day in reduced chemical mechanisms than the MCM~v3.2 leading to lower amounts of larger-sized degradation products that can further degrade and produce \ce{O_x}.
Thus, many VOC in reduced chemical mechanisms produce a lower maximum of \ce{O_x} than the MCM~v3.2 ultimately leading to lower \ce{O3} mixing ratios from the reduced chemical mechanisms compared to the MCM~v3.2.

Reactive VOC, such as unsaturated aliphatic and aromatic VOC, produce maximum \ce{O_x} on the first day of the simulations.
Unsaturated aliphatic VOC produce similar amounts of \ce{O_x} on the first day between mechanisms; differences in \ce{O_x} production arise when mechanism species are used to represent individual VOC.
Large inter-mechanism differences in \ce{O_x} production result from the degradation of aromatic VOC on the first day due to the faster break down of the mechanism species representing aromatic VOC in reduced chemical mechanisms.

The less-reactive alkanes produce maximum \ce{O_x} on the second day of simulations and this maximum is lower in each reduced chemical mechanism than the MCM~v3.2 due to the faster break down of alkanes into smaller sized degradation products on the first day.
The lower maximum in \ce{O_x} production during alkane degradation in reduced mechanisms would lead to an underestimation of the \ce{O3} levels downwind of VOC emissions, and an underestimation of the VOC contribution to tropospheric background \ce{O3} when using reduced mechanisms in regional or global modelling studies.

\section{Paper 2: }

\section{Paper 3: }
