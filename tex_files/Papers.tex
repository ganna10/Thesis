This chapter outlines the main findings in each scientific paper published as part of this thesis.
These publications addressed the research questions framed in Sect.~\ref{s:research_questions} and are found in Chaps.~\ref{c:paper_1}--\ref{c:paper_3}.

\singlespacing
\section[Paper I]{Paper I: A comparison of chemical mechanisms using tagged ozone production potential (TOPP) analysis} \label{s:chemical_mechanism_results}

\onehalfspacing

\noindent
Published: \bibentry{Coates:2015}.
%\vspace{5mm}

The first paper presents a box modelling study where the effects on ozone production from VOC degradation chemistry in reduced chemical mechanisms (Table~\ref{t:mechanisms}) were compared to the effects in the detailed MCM~v3.2 chemical mechanism.
This chemical mechanism comparison used the tagging approach described in Sect.~\ref{s:tagging} to give insights into how the simplified representation of VOC degradation by chemical mechanisms influenced maximum ozone production.

The difference in peak ozone between all chemical mechanisms on the first day was $21$~ppbv when including the outlier RACM chemical mechanism and $8$~ppbv when not including RACM.
The ozone mixing ratios using reduced chemical mechanisms, except CRI~v2 and RADM2, were lower than the ozone mixing ratios using the MCM~v3.2 on the first two days of simulations.
The representation of the degradation of aromatic VOC in RACM led to the lower ozone mixing ratios than all other chemical mechanisms.

The VOC degradation described in CRI~v2, a lumped-intermediate chemical mechanism, produced the most similar amounts of \ce{O_x} to the MCM~v3.2 for each VOC.
On the other hand, the degradation of VOC in all other reduced chemical mechanisms led to differences in \ce{O_x} production with the largest differences occuring after the first day of simulations.
The degradation of aromatic VOC in the reduced chemical mechanisms led to the largest differences in \ce{O_x} production from the MCM~v3.2.

Many VOC are broken down into smaller-sized degradation products faster on the first day in reduced chemical mechanisms than the MCM~v3.2.
The faster breakdown of VOC leads to lower amounts of larger-sized degradation products that can further degrade and produce \ce{O_x} in the reduced chemical mechanisms.
Thus, many VOC in reduced chemical mechanisms produce a lower maximum of \ce{O_x} than the MCM~v3.2 resulting in lower ozone mixing ratios from the reduced chemical mechanisms compared to the MCM~v3.2.

Reactive VOC, such as alkenes and aromatic VOC, produce maximum \ce{O_x} on the first day of the simulations.
Alkenes produce similar amounts of \ce{O_x} on the first day between chemical mechanisms; differences in \ce{O_x} production arise when mechanism species are used to represent individual VOC.
Large inter-mechanism differences in \ce{O_x} production result from the degradation of aromatic VOC on the first day due to the faster break down of the mechanism species representing aromatic VOC in reduced chemical mechanisms.
The less-reactive alkanes produce peak \ce{O_x} on the second day of simulations and this peak is lower in each reduced chemical mechanism than the MCM~v3.2 due to the faster break down of alkanes into smaller sized degradation products on the first day.

\singlespacing
\section[Paper II]{Paper II: Variation of the NMVOC Speciation in the Solvent Sector and the Sensitivity of Modelled Tropospheric Ozone} \label{s:EI_results}

\onehalfspacing

\noindent
Published: \bibentry{vonSchneidemesser:2016}.
%\vspace{5mm}

The second publication compared ozone prodution when using different emission inventories (EIs) of NMVOC emissions for the solvent sector within a box model.
The MCM~v3.2, MOZART-4 and RADM2 chemical mechanisms were used to ascertain whether the representation of tropospheric chemistry affected the differences in ozone production when using the different emission inventories.
Simulations using the tagged approach of Sect.~\ref{s:chemical_mechanism_results} were performed to allocate \ce{O_x} production to the emitted NMVOC specified by each EI.

A maximum difference in peak ozone mixing ratios ranged between $11$ and $15$~ppbv using different EIs with a single chemical mechanism.
When using the same EI, a maximum difference of $7$~ppbv in ozone mixing ratios was determined between simulations with different chemical mechanisms.
Thus both the choice of chemical mechanism and EI influenced the amount of ozone produced.

Further simulations using emissions from all other non-solvent sectors while varying the emissions from the solvent sector produced a lower maximum difference in ozone mixing ratio ($6$ -- $9$~ppbv) for each chemical mechanism.
Including emissions from biogenic sources further reduced the maximum differences in ozone mixing ratios even further ($5$ -- $8$~ppbv) with each chemical mechanism.

Reactive VOC, such as alkenes and aromatic VOC, contributed the most to \ce{O_x} production on the first day.
While less-reactive VOC, such as alkanes and oxygenated VOC, contributed the most to the cumulative \ce{O_x} production after seven days.  
A positive correlation between \ce{O_x} production and the contribution of alkane species by the EIs was determined while a negative correlation was determined between cumulative \ce{O_x} production and the contribution of oxygenated species by the EIs.
No correlation was found between the specification of aromatic species by EIs and \ce{O_x} production and not all EIs specify alkene emissions thus no correlation was made between alkene emissions and \ce{O_x} production.

\singlespacing
\section[Paper III]{Paper III: The Influence of Temperature on Ozone Production under varying \ce{NO_x} Conditions -- a modelling study} \label{s:T-O3_results}

\onehalfspacing

\noindent
Published: \bibentry{Coates:2016}.
%\vspace{5mm}

The final publication looked at the ozone-temperature relationship with different \ce{NO_x} conditions simulated by a box model.
A series of box model simulations varying the temperature and NO conditions were performed using NMVOC emissions representative of central Europe, first using a temperature-independent source of isoprene emissions followed by simulations using a temperature-dependent source of isoprene emissions.
All simulations were repeated using the MCM~v3.2, CRI~v2, MOZART-4, RADM2 and CB05 chemical mechanisms.

A non-linear relationship between ozone, temperature and \ce{NO_x} was produced using each chemical mechanism. 
This non-linear relationship was similar to that previously reported from observational studies.  
With each chemical mechanism, the absolute increase in ozone with temperature was greater for temperature-dependent chemistry than the increase ozone with temperature due to isoprene emissions.
The largest increases in ozone mixing ratios were obtained in High-\ce{NO_x} conditions and the lowest increase in ozone mixing ratios was achieved using Low-\ce{NO_x} conditions.

The \ce{O_x} production normalised by the total loss rate of emitted NMVOC was constant with temperature showing that the production of \ce{O_x} with temperature was controlled by the loss rate of VOCs.
The increased loss rate of VOCs with temperature was mainly due to the increase of OH with temperature. 
Net production of \ce{O_x} increased with temperature in all \ce{NO_x} conditions due the temperature dependent chemistry of \ce{RO2NO2} species, such as PAN.
At higher temperatures the equilibrium of \ce{RO2NO2} \eqref{r:RO2_NO2} shifts towards decomposition to \ce{RO2} and \ce{NO2} thus at higher temperatures more \ce{RO2} is available for \ce{O_x} prodution via \eqref{r:RO2_NOb}.

The box model results were also compared to observational data and output from the 3-D model WRF-Chem.
The rate of increase of ozone with temperature from the box model was about half the rate of increase of ozone with temperature using observational and WRF-Chem output.
Observational data and the output of 3-D models such as WRF-Chem include additional effects of temperature on ozone, such as stagnant conditions, where low wind speeds lead to the accumulation of oxidants promoting ozone production.
Box model simulations were performed without mixing to approximate stagnant conditions and in these box model simulations the increase in ozone with temperature was faster than the box model simulations including mixing.
The differences in the rate of increase of ozone with temperature between mixing simulations and chemical mechanisms indicated a higher sensitivity to mixing than the choice of chemical mechanism.
