This chapter will outline the main findings in each of the scientific papers that were published as part of the PhD.

\section{Paper 1: A comparison of chemical mechanisms using tagged ozone production potential (TOPP) analysis}

Published: \bibentry{Coates:2015}.
\vspace{5mm}

The first paper was a box modelling study in which the secondary chemistry represented in the reduced chemical mechanisms in Table \todo{table} for the VOC in Table \todo{another table} were compared to the detailed MCM chemical mechanisms.
The research question of this paper was to verify whether these different representations of secondary chemistry influence ozone production.

Each chemical mechanism in Table \todo{table} was ``tagged'' for each VOC in Table \todo{another table}. 
Tagging the chemical mechanisms involved labelling each organic degradation product from a particular VOC degradation with the name of the emitted VOC, thus each VOC has a separate set of reactions describing its degradation until the final products (\ce{CO2} and \ce{H2O}) are produced.
The advantage of using the tagging approach is that \ce{O_x} production, used as a proxy for \ce{O3} production, can be attributed to the individual VOC sources.

In general, the ozone mixing ratios obtained from the model runs using reduced chemical mechanisms was lower than the mixing ratios from the reference MCM chemical mechanisms.
The CRI~v2, a lumped-intermediate mechanism, produced the most similar amounts of \ce{O_x} to the MCM~v3.2 for each VOC.
Thus, the technique of using lumped-intermediate species whose degradation are based upon more detailed chemical mechanisms is preferable for the development of chemical mechanisms in the future.

Reactive VOC, such as aromatics and unsaturated aliphatic VOC, produce maximum \ce{O_x} on the first day of the simulations and this is also when the largest inter-mechanism differences arise for most of these VOC.
Unsaturated aliphatic VOC produce similar amounts of \ce{O_x} on the first day between mechanisms; differences in \ce{O_x} production arises when mechanism species are used to represent the particular VOC.
Larger differences in first day \ce{O_x} production between chemical mechanisms is achieved for aromatic VOC, as many aromatic VOC are represented by a mechanism species that does not accurately describe the \ce{O_x} production from the VOC.

The less-reactive alkanes produce maximum \ce{O3} on the second day of simulations and this maximum is lower in each reduced chemical mechanism than the MCM~v3.2 due to the faster break down of alkanes into smaller sized degradation products on the first day.
The lower maximum in \ce{O3} production during alkane degradation in reduced mechanisms would lead to an underestimation of the \ce{O3} levels downwind of VOC emissions, and an underestimation of the VOC contribution to tropospheric background \ce{O3} when using reduced mechanisms in regional or global modelling studies.

Many VOC are broken down into smaller sized degradation products faster on the first day in reduced chemical mechanisms than the MCM~v3.2 leading to lower amounts of larger sized degradation products that can further degrade and produce \ce{O_x}.
Thus, many VOC in reduced chemical mechanisms produce a lower maximum of \ce{O_x} than the MCM~v3.2 ultimitly leading to lower \ce{O3} mixing ratios from the reduced chemical mechanisms compared to the MCM~v3.2.

\section{Paper 2: }

\section{Paper 3: }
