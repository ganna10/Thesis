This chapter outlines the main findings in each scientific paper published as part of this thesis.
These publications, found in Chaps.~\ref{c:paper_1}--\ref{c:paper_3}, addressed the research questions framed in Sect.~\ref{s:research_questions}.

\singlespacing
\section[Paper I]{Paper I: A comparison of chemical mechanisms using tagged ozone production potential (TOPP) analysis} \label{s:chemical_mechanism_results}

\onehalfspacing

\noindent
Published: \bibentry{Coates:2015}.
%\vspace{5mm}

\todo[noline]{decompose sentence to make clearer}
This paper compared the effects of VOC degradation on ozone production described by the reduced chemical mechanisms listed in Table~\ref{t:mechanisms} to the detailed MCM~v3.2 chemical mechanism.
This chemical mechanism comparison used the tagging approach described in Sect.~\ref{s:tagging} to obtain insights into the influences of the simplified representations of VOC degradation by chemical mechanisms on ozone production.

A comparison of the time series of ozone mixing ratios on the first two days of simulations showed that they were generally lower using the reduced chemical mechanisms than using the MCM~v3.2.
%The time series of ozone mixing ratios on the first two days of simulations using reduced chemical mechanisms were generally lower than the time series of ozone mixing ratios using the MCM~v3.2.
The difference in peak ozone on the first day between all chemical mechanisms was $21$~ppbv when including the outlier RACM chemical mechanism and $8$~ppbv when not including RACM.
The representation of the degradation of aromatic VOC in RACM led to lower ozone mixing ratios than all other chemical mechanisms.

The VOC degradation described in CRI~v2, a lumped-intermediate chemical mechanism, produced the most similar amounts of \ce{O_x} to the MCM~v3.2 for each VOC.
On the other hand, the VOC degradation in all other reduced chemical mechanisms led to differences in \ce{O_x} production with the largest differences occuring after the first day of simulations.
The degradation of aromatic VOC in the reduced chemical mechanisms led to the largest differences in \ce{O_x} production from the MCM~v3.2.

Many VOC were broken down into smaller-sized degradation products faster on the first day in reduced chemical mechanisms than the MCM~v3.2.
The faster breakdown of VOC leads to lower amounts of larger-sized degradation products that may further degrade and produce \ce{O_x}.
Thus, many VOC in reduced chemical mechanisms produced a lower peak of \ce{O_x} than the MCM~v3.2 leading to lower ozone mixing ratios from the reduced chemical mechanisms compared to the MCM~v3.2.

Reactive VOC, such as alkenes and aromatic VOC, produced peak \ce{O_x} on the first day of simulations.
Alkenes produced similar amounts of \ce{O_x} on the first day between chemical mechanisms with differences in \ce{O_x} production arising when mechanism species represented individual VOC.
Large inter-mechanism differences in \ce{O_x} production resulted from the degradation of aromatic VOC on the first day due to the faster break down of the mechanism species representing aromatic VOC in reduced chemical mechanisms.
The less-reactive alkanes produced peak \ce{O_x} on the second day of simulations with peak \ce{O_x} lower in each reduced chemical mechanism than the MCM~v3.2 due to the faster break down of alkanes into smaller sized degradation products on the first day by the reduced chemical mechanisms.

\singlespacing
\section[Paper II]{Paper II: Variation of the NMVOC Speciation in the Solvent Sector and the Sensitivity of Modelled Tropospheric Ozone} \label{s:EI_results}

\onehalfspacing

\noindent
Published: \bibentry{vonSchneidemesser:2016}.
%\vspace{5mm}

The second publication compared ozone production using different emission inventories (EIs) of NMVOC emissions for the solvent sector.
The MCM~v3.2, MOZART-4 and RADM2 chemical mechanisms were used to ascertain whether the representation of tropospheric chemistry affected the differences in ozone production when using the different solvent sector EIs.
Simulations using the tagged approach were performed to allocate \ce{O_x} production to the emitted NMVOC specified by each EI.

A maximum difference in peak ozone mixing ratios ranged between $11$ and $15$~ppbv with different EIs using a single chemical mechanism.
When using the same EI, a maximum difference of $7$~ppbv in peak ozone mixing ratios was determined between simulations with different chemical mechanisms.
Thus both the choice of chemical mechanism and EI influenced ozone production.

A lower maximum difference in peak ozone ($6$ -- $9$~ppbv) was produced from simulations using emissions from all other non-solvent emission sectors while varying the emissions from the solvent sector with each chemical mechanism.
Including emissions from biogenic sources further reduced the maximum differences in peak ozone mixing ratios ($5$ -- $8$~ppbv) with each chemical mechanism.

Reactive VOC, such as alkenes and aromatics, contributed the most to \ce{O_x} production on the first day.
While less-reactive VOC, such as alkanes and oxygenated VOC, contributed the most to the cumulative \ce{O_x} production after seven days of simulations. 

A positive correlation was determined between \ce{O_x} production and the contribution of alkane species by the EI.
While a negative correlation was determined between cumulative \ce{O_x} production and the contribution of oxygenated species by the EIs.
No correlation was found between the specification of aromatic species by EIs and \ce{O_x} production and as not all solvent sector EIs specify alkene emissions no correlation was calculated between specified alkene emissions and \ce{O_x} production.

\singlespacing
\section[Paper III]{Paper III: The Influence of Temperature on Ozone Production under varying \ce{NO_x} Conditions -- a modelling study} \label{s:T-O3_results}

\onehalfspacing

\noindent
Submitted: \bibentry{Coates:2016}.
%\vspace{5mm}

The final publication looked at the ozone-temperature relationship with different \ce{NO_x} conditions simulated by a box model.
A series of box model simulations varying temperature and \ce{NO_x} conditions were performed using NMVOC emissions representative of central Europe, first using a temperature-independent source of isoprene emissions followed by simulations using a temperature-dependent source of isoprene emissions.
All simulations were repeated using the MCM~v3.2, CRI~v2, MOZART-4, RADM2 and CB05 chemical mechanisms.

Each chemical mechanism produced a non-linear relationship between ozone, temperature and \ce{NO_x}.
This non-linear relationship was similar to that previously determined by observational studies.  
With each chemical mechanism, the absolute increase in ozone with temperature was slightly higher for temperature-dependent chemistry than the increase in ozone with temperature due to isoprene emissions.
The largest increases in ozone mixing ratios with temperature were obtained with moderate \ce{NO_x} conditions and the lowest increase in ozone mixing ratios was achieved with low \ce{NO_x} conditions.

The \ce{O_x} production normalised by the total loss rate of emitted NMVOC was roughly constant with temperature showing that the production of \ce{O_x} with temperature was controlled by the loss rate of VOCs.
The increased loss rate of VOCs with temperature was mainly due to the increased OH-reactivity of the emitted NMVOC with temperature. 
Net production of normalised \ce{O_x} increased with temperature in all \ce{NO_x} conditions due the temperature dependent chemistry of \ce{RO2NO2} species.
At higher temperatures the equilibrium of \ce{RO2NO2} \eqref{r:RO2_NO2} shifts towards decomposition to \ce{RO2} and \ce{NO2} thus at higher temperatures more \ce{RO2} is available to produce \ce{O_x} via \eqref{r:RO2_NOb}.

The box model results were also compared to observational data and output from the WRF-Chem model.
The rate of increase of ozone with temperature from the box model was about half the rate of increase of ozone with temperature using observational and WRF-Chem output.
Observational data and the output of 3-D models such as WRF-Chem include additional effects of temperature on ozone such as stagnantion where low wind speeds lead to the accumulation of reactants promoting ozone production.
Box model simulations without mixing were performed to approximate stagnant conditions which increased the rate of increase in ozone with temperature than the box model simulations including mixing.
These results indicated that the modelled relationship between ozone, temperature and \ce{NO_x} was more sensitive to mixing than the choice of chemical mechanism.
